\documentclass[11pt]{article}
\usepackage{amssymb}
\usepackage{amsthm}
\usepackage{enumitem}
\usepackage{amsmath}
\usepackage{bm}
\usepackage{adjustbox}
\usepackage{mathrsfs}
\usepackage{graphicx}
\usepackage{siunitx}
\usepackage{physics}
\usepackage[mathscr]{euscript}

\title{\textbf{Solved selected problems of Analytical Mechanics by Nivaldo Lemos}}
\author{Franco Zacco}
\date{}

\addtolength{\topmargin}{-3cm}
\addtolength{\textheight}{3cm}

\newcommand{\R}{\mathbb{R}}
\newcommand{\C}{\mathbb{C}}
\newcommand{\hatr}{\bm{\hat{r}}}
\newcommand{\hatx}{\bm{\hat{x}}}
\newcommand{\haty}{\bm{\hat{y}}}
\newcommand{\hatz}{\bm{\hat{z}}}
\newcommand{\hatth}{\bm{\hat{\theta}}}
\newcommand{\hatphi}{\bm{\hat{\phi}}}
\newcommand{\hatrho}{\bm{\hat{\rho}}}
\newcommand{\ei}[1]{\vec{\bm{e}}_#1}
\theoremstyle{definition}
\newtheorem*{solution*}{Solution}
\renewcommand*{\proofname}{\textbf{Solution}}

\begin{document}
\maketitle
\thispagestyle{empty}

\begin{proof}{\textbf{1.1}}
Let
\begin{align*}
    (x^2 + y^2)dx + xzdz = 0 \qquad (x^2 + y^2)dy + yzdz = 0
\end{align*}
be constraints.
\begin{itemize}
\item[(a)] We have two constraints that can be expressed as 1-forms
$\omega^1 = 0$ and $\omega^2 = 0$ where
\begin{align*}
    \omega^1 = (x^2 + y^2)dx + xzdz  \qquad \omega^2 = (x^2 + y^2)dy + yzdz
\end{align*}
Also, $d\omega^1$ and $d\omega^2$ are
\begin{align*}
    d\omega^1 &= 2ydy\wedge dx + zdx\wedge dz\quad\text{and}\quad
    d\omega^2 = 2xdx\wedge dy + zdy\wedge dz
\end{align*}
So, let us consider them separately, i.e. we want to compute
$d\omega^1 \wedge \omega^1$ and $d\omega^2 \wedge \omega^2$ as follows
\begin{align*}
    d\omega^1 \wedge \omega^1
    &= (2ydy\wedge dx + zdx\wedge dz) \wedge ((x^2 + y^2)dx + xzdz)\\
    &= 2xyz~dy\wedge dx \wedge dz
\end{align*}
And
\begin{align*}
    d\omega^2 \wedge \omega^2
    &= (2xdx\wedge dy + zdy\wedge dz) \wedge ((x^2 + y^2)dy + yzdz)\\
    &= 2xyz~dx\wedge dy \wedge dz
\end{align*}
We see that $d\omega^1 \wedge \omega^1 \neq 0$ unless $x=y=z=0$ and the
same happens for $d\omega^2 \wedge \omega^2$.

Therefore, they are not integrable when considered separately.

\item[(b)]
The 2-form $\Omega$ defined by $\Omega = \omega^1 \wedge \omega^2$ is given by
\begin{align*}
    \Omega &= (x^2 + y^2)dx + xzdz \wedge (x^2 + y^2)dy + yzdz\\
    &= (x^2 + y^2)dx \wedge (x^2 + y^2)dy + (x^2 + y^2)dx \wedge yzdz\\
    &\quad+ xzdz \wedge (x^2 + y^2)dy\\
    &= (x^2 + y^2)^2 dx \wedge dy + yz(x^2 + y^2) dx \wedge dz
    + xz(x^2 + y^2) dz \wedge dy
\end{align*}
It follows that
\begin{align*}
    d\omega^1 \wedge \Omega
    &= 2y(x^2 + y^2)^2 dy \wedge dx \wedge dx\wedge dy
    + 2y^2z(x^2 + y^2) dy\wedge dx \wedge dx \wedge dz\\
    &\quad+ 2yxz(x^2 + y^2) dy\wedge dx \wedge dz \wedge dy
    + z(x^2 + y^2)^2 dx \wedge dz \wedge dy \wedge dx\\
    &\quad+ yz^2(x^2 + y^2) dx\wedge dz \wedge dx \wedge dz
    + xz^2(x^2 + y^2) dx\wedge dz \wedge dz \wedge dy\\
    &= 0
\end{align*}
And in the same way $d\omega^2 \wedge \Omega = 0$.

Therefore, the constraints are integrable when considered together.

On the other hand, let $x,y,z \neq 0$ then we see that
\begin{align*}
    d \ln\frac{y}{x} = \frac{1}{y}dy-\frac{1}{x}dx
\end{align*}
Also, let us consider the following 
\begin{align*}
    y\omega^1 - x\omega^2 &= y(x^2 + y^2)dx + xyzdz - x(x^2 + y^2)dy - xyzdz\\
    &= (x^2 + y^2)(ydx - xdy)
\end{align*}
But since $\omega^1 = \omega^2 = 0$ then must be that
\begin{align*}
    ydx - xdy &= 0\\
    \frac{1}{y}dy - \frac{1}{x}dx &= 0
\end{align*}
Then the constraints are equivalent to $d \ln(y/x) = 0$.

Now, considering the combination
\begin{align*}
    x\omega^1 + y\omega^2 &= x(x^2 + y^2)dx + x^2zdz + y(x^2 + y^2)dy + y^2zdz\\
    &= (x^2 + y^2)(xdx + ydy + zdz)
\end{align*}
Must be that $xdx + ydy + zdz = 0$ but we see that
\begin{align*}
    d(x^2 + y^2 + z^2) = xdx + ydy + zdz
\end{align*}
Therefore, the constraints are equiavalent also to $xd(x^2 + y^2 + z^2) = 0$.

\cleardoublepage
\item[(c)] Finally, integrating the equations
\begin{align*}
    xd(x^2 + y^2 + z^2) = 0 \quad\text{and}\quad d\ln\frac{y}{x} = 0
\end{align*}
We get that
\begin{align*}
    \int d(x^2 + y^2 + z^2) &= 0\\
    x^2 + y^2 + z^2 + C_1' &= 0\\
    x^2 + y^2 + z^2 &= C_1
\end{align*}
And
\begin{align*}
    \int d \ln\frac{y}{x} &= 0\\
   \ln\frac{y}{x} + C_2' &= 0\\
   \frac{y}{x} &= 1 - e^{C_2'}\\
   y &= C_2 x
\end{align*}
Where we renamed $C_1 = -C_1'$ and $C_2 = 1 - e^{C_2'}$.
\end{itemize}
\end{proof}

\cleardoublepage
\begin{proof}{\textbf{1.2}}
Let us consider the system from Example 1.18 in the case where
$m_1 = m_2 = m_3 = m$ then the equations of motion become
\begin{align*}
    m\ddot{x}_2 - \frac{k}{2}(x_3 - x_2 - l) &= 0\\
    m\ddot{x}_3 - mg + k(x_3 - x_2 -l) &= 0
\end{align*}
Let us define $y = x_3 - x_2 - l$ then
\begin{align*}
    \ddot{y} &= \ddot{x}_3 - \ddot{x}_2
\end{align*}
Then, subtracting the first equation from the second one we get that
\begin{align*}
    m\ddot{x}_3 + k(x_3 - x_2 -l) - m\ddot{x}_2 + \frac{k}{2}(x_3 - x_2 - l) &= mg\\
    (\ddot{x}_3 - \ddot{x}_2) + \frac{3k}{2m}(x_3 - x_2 - l)&= g\\
    \ddot{y} + \frac{3k}{2m}y &= g
\end{align*}
We know that in terms of $y$ the solution to this equation is
\begin{align*}
    y(t) = C_1\sin(\sqrt{\frac{3k}{2m}}t) + C_2\cos(\sqrt{\frac{3k}{2m}}t) 
    + \frac{2gm}{3k}
\end{align*}
or letting $\omega = \sqrt{\frac{3k}{2m}}$ we get that
\begin{align*}
    y(t) = C_1\sin(\omega t) + C_2\cos(\omega t) + \frac{2mg}{3k}
\end{align*}
Using the initial conditions $x_2 = 0$ and $x_3 = l$ we can determine $C_2$.
We see that $y(t=0) = l - 0 - l = 0$ then $C_2 = -\frac{2mg}{3k}$.
\\
On the other hand,
\begin{align*}
    \dot{y}(t) = C_1\omega\cos(\omega t) - C_2\omega \sin(\omega t)
\end{align*}
And since the system starts from rest i.e. $\dot{y} = 0$ we get that 
$C_1 = 0$. Therefore the equation becomes
\begin{align*}
    y(t) = \frac{2mg}{3k}(1 - \cos(\omega t)) = x_3 - x_2 - l
\end{align*}
So, replacing in the first original differential equation we get that
\begin{align*}
    \ddot{x}_2 + \frac{g}{3}(\cos(\omega t) -1) = 0
\end{align*}
Integrating with respect to $t$ gives us
\begin{align*}
    \dot{x}_2 &= -\frac{g}{3} \int (\cos(\omega t) -1)~dt =
    \frac{g}{3\omega}\bigg[\omega t-\sin(\omega t)\bigg] + C
\end{align*}
Where $C = 0$ since $\dot{x}_2 = 0$ at $t = 0$. Integrating again
\begin{align*}
    x_2 &= \frac{g}{3}\int\bigg[-\frac{1}{\omega}\sin(\omega t) + t\bigg]~dt\\
    &= \frac{g}{3}\frac{\cos(\omega t)}{\omega^2} + \frac{g}{6}t^2 + C\\
    &= \frac{2mg}{9k}\cos(\omega t) + \frac{g}{6}t^2 + C
\end{align*}
Where we replaced the value of $\omega$. Now since $x_2 = 0$ when $t = 0$ we
have that $C = -2mg/9k$, so, finally
\begin{align*}
    x_2 = \frac{2mg}{9k}(\cos(\omega t) - 1) + \frac{g}{6}t^2
\end{align*}
From the equation we got for $\dot{x}_2$, let $t = \pi/2\omega$, at this
moment $\sin(\omega t) = 1$ which is as negative as $-\sin(\omega t)$ can get.
Hence
\begin{align*}
    \omega t - \sin(\omega t) = \frac{\pi}{2} - 1 
\end{align*}
But since $\pi/2 > 1$, then, $\dot{x}_2 > 0$ at this moment. For any point later,
$t$ will grow even more and hence $\omega t - \sin(\omega t)$ will be positive.
\\
Therefore if $t > 0$ then $\dot{x}_2 > 0$.
\end{proof}

\cleardoublepage
\begin{proof}{\textbf{1.5}}
Let us consider Atwood's machine shown in Fig. 1.10. The string connecting
$M$ and $m$ must be inextensible, so, let $x$ be the vertical coordinate
which determines the position of $M$, then must be that $x + c + r = l$ where $l$
is the constant length of the string and $c$ is the length between pulleys.
\\
The kinetic energy experienced by $M$ is then $T_M = M\dot{x}^2/2 = M\dot{r}^2/2$.
\\
So the kinetic energy of the system is
\begin{align*}
    T = \frac{M}{2}\dot{r}^2 + \frac{m}{2}(\dot{r}^2 + r^2\dot{\theta}^2)
    = \frac{m + M}{2}\dot{r}^2 + \frac{m}{2}r^2\dot{\theta}^2
\end{align*}
On the other hand, for the potential energy since the string is inextensible
then any change in the $r$ coordinate will imply a change in potential energy
for the $M$ mass so we can write that
\begin{align*}
    V = Mgr - mgr\cos\theta = gr(M - m\cos\theta)
\end{align*}
Therefore, the Lagrangian is
\begin{align*}
    L = T - V
    = \frac{m + M}{2}\dot{r}^2 + \frac{m}{2}r^2\dot{\theta}^2
    - gr(M - m\cos\theta)
\end{align*}
Now, we want to write Lagrange's equations for the system. We compute
first the derivatives
\begin{align*}
    \pdv{L}{\dot{r}} &= (m + M)\dot{r} &
    \pdv{L}{r} &= mr\dot{\theta}^2 - g(M -m\cos\theta)\\
    \pdv{L}{\dot{\theta}} &= mr^2\dot{\theta} &
    \pdv{L}{\theta} &= -mgr\sin\theta
\end{align*}
Then the Lagrange's equations are
\begin{align*}
    \dv{t}(\pdv{L}{\dot{r}}) - \pdv{L}{r}
    &= (m + M)\ddot{r} - mr\dot{\theta}^2 + g(M -m\cos\theta) = 0\\
    \dv{t}(\pdv{L}{\dot{\theta}}) - \pdv{L}{\theta}
    &= mr(2\dot{\theta}\dot{r} + r\ddot{\theta}) + mgr\sin\theta = 0
\end{align*}
\end{proof}

\cleardoublepage
\begin{proof}{\textbf{1.6}}
Let us consider Huygens' cycloidal pendulum with the following parametric
equations
\begin{align*}
    x = R(\theta - \sin\theta) \qquad y = R(1 - \cos\theta)
\end{align*}
where the vertical $y$-axis points downward.
\\
Then the kinetic energy of the system is given by
\begin{align*}
    T &= \frac{m}{2}(\dot{x}^2 + \dot{y}^2)\\
    &= \frac{m}{2}R^2[(\dot\theta - \dot{\theta}\cos\theta)^2 + \dot{\theta}^2\sin^2\theta]\\
    &= \frac{m}{2}R^2[\dot\theta^2 - 2\dot\theta^2\cos\theta
    + \dot\theta^2\cos^2\theta + \dot\theta^2\sin^2\theta]\\
    &= \frac{m}{2}R^2[1 - 2\cos\theta + 1]\dot\theta^2\\
    &= 2mR^2\bigg[\frac{1 - \cos\theta}{2}\bigg]\dot\theta^2\\
    &= 2mR^2\sin^2\bigg(\frac{\theta}{2}\bigg)\dot\theta^2
\end{align*}
Where we used the trigonometric identity $\sin^2(\theta/2) = (1 - \cos\theta)/2$.
\\
On the other hand, the potential energy only depends on $y$ and since the
$y$-axis points downward we have that
\begin{align*}
    V = -mgy =-mgR(1-\cos\theta)
\end{align*}
Therefore, the Lagrangian of the system is
\begin{align*}
    L = T - V = 2mR^2\sin^2\bigg(\frac{\theta}{2}\bigg)\dot\theta^2 + mgR(1-\cos\theta)
\end{align*}
Let now, $u = \cos(\theta/2)$ then $\theta = 2\arccos(u)$, so the Lagrangian in
terms of $u$ is given by
\begin{align*}
    L &= 2mR^2\sin^2(\arccos(u))\bigg(-\frac{2\dot{u}}{\sqrt{1 - u^2}}\bigg)^2
    + mgR(1-\cos(2\arccos(u)))\\
    L &= 2mR^2(1 - u^2)\bigg(\frac{4\dot{u}^2}{1 - u^2}\bigg)
    + mgR(1 - 2\cos^2(\arccos(u)) + 1)\\
    L &= 8mR^2\dot{u}^2 + 2mgR(1 - u^2)
\end{align*}
Then the Lagrange equation in terms of $u$ is
\begin{align*}
    \dv{t}(\pdv{L}{\dot u}) - \pdv{L}{u} = 16mR^2\ddot{u} + 4mgRu = 0
\end{align*}
Finally, we can write this equation as
\begin{align*}
    4R^2\ddot{u} + gRu &= 0\\
    \ddot{u} &= -\frac{g}{4R}u
\end{align*}
This is the equation of a simple harmonic oscillator where $\omega^2 = g/4R$,
or in terms of the period of oscillation $\omega = 2\pi/T$ then we get that
\begin{align*}
    \frac{4\pi^2}{T^2} &= \frac{g}{4R}\\
    T^2 &= \frac{16\pi^2R}{g}\\
    T &= 4\pi\sqrt{\frac{R}{g}}
\end{align*}
\end{proof}

\cleardoublepage
\begin{proof}{\textbf{1.7}}
Let a projectile fired near the surface of the earth. Taking the Cartesian
coordinates as the generalized coordinates we have that the kinetic energy is
\begin{align*}
    T = \frac{1}{2}mv^2 = \frac{1}{2}m(v_x^2 + v_y^2)
\end{align*}
And the potential energy is 
\begin{align*}
    V = mgy
\end{align*}
Therefore the lagrangian is given by
\begin{align*}
    L = T - V = \frac{1}{2}m(v_x^2 + v_y^2) - mgy
\end{align*}
We solve then Lagrange's equation to get the equations of motion as follows
\begin{align*}
    \dv{t}(\pdv{L}{v_x}) - \pdv{L}{x} + \pdv{\mathcal{F}}{v_x} &= 
    m\dot{v}_x + \lambda v_x = 0\\
    \dv{t}(\pdv{L}{v_y}) - \pdv{L}{y} + \pdv{\mathcal{F}}{v_y} &= 
    m\dot{v}_y + mg + \lambda v_y = 0
\end{align*}
Then the equations of motion are
\begin{align*}
    \ddot{x} + \frac{\lambda}{m}\dot{x} = 0
    \qquad
    \ddot{y} + \frac{\lambda}{m}\dot{y} = -g
\end{align*}
\end{proof}

\cleardoublepage
\begin{proof}{\textbf{1.8}}
\begin{itemize}
\item [(a)] Let us define the following lagrangian
\begin{align*}
    L = e^{\lambda t/m}\bigg[\frac{m}{2}(\dot{x}^2 + \dot{y}^2) - mgy\bigg]
\end{align*}
Lagrange's equation for the $x$ coordinate is
\begin{align*}
    \dv{t}(\pdv{L}{\dot{x}}) - \pdv{L}{x} &= 0\\
    \dv{t}(e^{\lambda t/m}m\dot{x}) &= 0\\
    me^{\lambda t/m}\ddot{x} + \lambda e^{\lambda t/m}\dot{x} &= 0\\
    \ddot{x} + \frac{\lambda}{m}\dot{x} &= 0
\end{align*}
And Lagrange's equation for the $y$ coordinate is
\begin{align*}
    \dv{t}(\pdv{L}{\dot{y}}) - \pdv{L}{y} &= 0\\
    \dv{t}(e^{\lambda t/m}m\dot{y}) + mge^{\lambda t/m} &= 0\\
    me^{\lambda t/m}\ddot{y} + \lambda e^{\lambda t/m}\dot{y}
    + mge^{\lambda t/m} &= 0\\
    \ddot{y} + \frac{\lambda}{m}\dot{y} &= -g
\end{align*}
\item [(b)] Let us solve the equations of motion now assuming the projectile is
fired from the origin with a velocity of magnitude $v_0$ making an angle
$\theta_0$ with the horizontal.

We see that the equation for $x$ can be solved by integration, first let us
rename $v = \dot{x}$ then the equation becomes
\begin{align*}
    \dv{v}{t} &= -\frac{\lambda}{m}v\\
    \int \frac{dv}{v} &= -\frac{\lambda}{m} \int dt\\
    \log(v) &= -\frac{\lambda}{m}t + C\\
    v &= c_1e^{-\lambda t /m}
\end{align*}
Now, replacing back again $v$ and integrating again we get that
\begin{align*}
    \int dx &= c_1 \int e^{-\lambda t /m} dt\\
    x &= -c_1\frac{m}{\lambda} e^{-\lambda t /m} + c_2
\end{align*}
We can determine the values of $c_1$ and $c_2$ by plugging in the initial
conditions. We know that $x(t=0) = 0$ hence $0 = c_2 - c_1m/\lambda$.

Also, we know that $\dot{x}(t=0) = v_0\cos\theta_0$ so since
$\dot{x} = c_1e^{-\lambda t/m}$ we get that $c_1 = v_0\cos\theta_0$, and using
this result we get that $c_2 = v_0m\cos\theta_0 /\lambda$.
Then the equation for $x$ becomes
\begin{align*}
    x = \frac{v_0\cos\theta_0 m}{\lambda}(1 - e^{-\lambda t /m})
\end{align*}

On the other hand, to solve the equation for $y$ first we solve the homogeneous
equation $\ddot{y} + \lambda \dot{y}/m = 0$ and then we add a particular
solution to get the general solution. The homogeneous equation has the same
form as the one we solved for $x$ so the solution is 
\begin{align*}
    y = c_2 -c_1\frac{m}{\lambda} e^{-\lambda t /m}
\end{align*}
Now, let us try the particular solution $y = -(mg/\lambda) t$ then
we see that $\dot{y} = -mg/\lambda$ and $\ddot{y} = 0$ so we get that
\begin{align*}
    0 + \frac{\lambda}{m}\bigg(-\frac{mg}{\lambda}\bigg) = -g
\end{align*}
So $y = -(mg/\lambda) t$ is a particular solution. Then the general solution is
\begin{align*}
    y = c_2 -c_1\frac{m}{\lambda} e^{-\lambda t /m} - \frac{mg}{\lambda} t
\end{align*}
When $t=0$ we get that $y(t=0) = 0$ then $c_2 = c_1m/\lambda$. Also, we 
have that $\dot{y} = c_1e^{-\lambda t/m} - mg/\lambda$ and when $t=0$
we know that $\dot{y} = v_0\sin\theta_0$ so
$c_1 = v_0\sin\theta_0 + mg/\lambda$ and hence $c_2 = v_0\sin\theta_0m/\lambda + m^2g/\lambda^2$.
Therefore the general solution becomes
\begin{align*}
    y = \bigg(v_0\sin\theta_0\frac{m}{\lambda} + \frac{m^2g}{\lambda^2}\bigg)
    (1 - e^{-\lambda t /m}) - \frac{mg}{\lambda} t
\end{align*}

\item [(c)] From the equation for $x$ we can get the time $t$ as follows
\begin{align*}
    1 - e^{-\lambda t/m} &= \frac{\lambda x}{m v_0\cos\theta_0}\\
    -\frac{\lambda t}{m} &= \log\bigg(1 -  \frac{\lambda x}{m v_0\cos\theta_0}\bigg)\\
    t &=- \frac{m}{\lambda}\log\bigg(1 -  \frac{\lambda x}{m v_0\cos\theta_0}\bigg)
\end{align*}
Now, replacing $t$ in the equation for $y$ we get the equation for the trajectory
of the projectile
\begin{align*}
    y &= \bigg(v_0\sin\theta_0\frac{m}{\lambda} + \frac{m^2g}{\lambda^2}\bigg)
    \frac{\lambda x}{m v_0\cos\theta_0} + \frac{m^2g}{\lambda^2}
    \log\bigg(1 -  \frac{\lambda x}{m v_0\cos\theta_0}\bigg)\\
    y &= \bigg(\tan\theta_0 + \frac{mg}{\lambda v_0\cos\theta_0}\bigg)x
    + \frac{m^2g}{\lambda^2}
    \log\bigg(1 -  \frac{\lambda x}{m v_0\cos\theta_0}\bigg)
\end{align*}

\end{itemize}
\end{proof}

\cleardoublepage
\begin{proof}{\textbf{1.9}}
\begin{itemize}
\item [(a)] Taking the generalized coordinate $x$ from the middle point when
both springs are at its natural length and $\theta$ measured from vertical
then the coordinates of the mass $m$ are
\begin{align*}
    x' &= x + l\sin\theta\\
    y' &= l\cos\theta
\end{align*}
So the velocities of $m$ are
\begin{align*}
    \dot{x'} &= \dot{x} + \dot{\theta} l\cos\theta\\
    \dot{y'} &= -\dot{\theta}l\sin\theta
\end{align*}
Then the kinetic energy is
\begin{align*}
    T &= \frac{1}{2}m((\dot{x} + \dot{\theta} l\cos\theta)^2 + (-\dot{\theta}l\sin\theta)^2)\\
    &= \frac{1}{2}m(\dot{x}^2 + 2\dot{x}\dot{\theta} l\cos\theta
    + \dot{\theta}^2 l^2\cos^2\theta + \dot{\theta}^2l^2\sin^2\theta)\\
    &= \frac{1}{2}m(\dot{x}^2 + 2\dot{x}\dot{\theta} l\cos\theta
    + \dot{\theta}^2l^2)
\end{align*}
And the potential energy is 
\begin{align*}
    V &= -mgl\cos\theta + \frac{1}{2}k(x^2 + x^2) = -mgl\cos\theta + kx^2
\end{align*}
Where we assumed that the natural length of the springs is 0 for simplicity.
\\
Then the Lagrangian is given by
\begin{align*}
    L = T - V = \frac{1}{2}m(\dot{x}^2 + 2\dot{x}\dot{\theta} l\cos\theta
    + \dot{\theta}^2l^2) + mgl\cos\theta - kx^2
\end{align*}
Now, we solve the Lagrange's equation as follows
\begin{align*}
    \dv{t}(\pdv{L}{\dot{x}}) - \pdv{L}{x} &= 0\\
    \dv{t}(m(\dot{x} + \dot{\theta} l\cos\theta)) + 2kx &= 0\\
    m(\ddot{x} + \ddot{\theta} l\cos\theta - \dot{\theta}^2l\sin\theta)
    + 2kx &= 0
\end{align*}
And
\begin{align*}
    \dv{t}(\pdv{L}{\dot{\theta}}) - \pdv{L}{\theta} &= 0\\
    \dv{t}(m(\dot{x}l\cos\theta + \dot{\theta} l^2))
    + m\dot{x}\dot{\theta}l\sin\theta + mgl\sin\theta&= 0\\
    m(\ddot{x}l\cos\theta - \dot{x}\dot{\theta}l\sin\theta + \ddot{\theta} l^2)
    + m\dot{x}\dot{\theta}l\sin\theta + mgl\sin\theta&= 0\\
    m(\ddot{x}l\cos\theta + \ddot{\theta} l^2) + mgl\sin\theta&= 0
\end{align*}

\cleardoublepage
\item [(b)] In the case of small angular oscillations $\sin\theta \approx \theta$
and $\cos\theta \approx 1$ and terms such as $\theta^n$ and $\dot\theta^n$ can
be neglected. Then the equations of motion become
\begin{align*}
    m(\ddot{x} + \ddot{\theta} l) + 2kx &= 0\\
    m(\ddot{x} + \ddot{\theta} l) + mg\theta &= 0
\end{align*}
Then subtracting both equations we see that
\begin{align*}
    2kx - mg\theta &= 0\\
    x &= \frac{mg}{2k}\theta
\end{align*}
So $x$ can be written as $x = \alpha \theta$ where $\alpha = mg/2k$.

Now, computing $\ddot{x} = \alpha \ddot{\theta}$ and replacing it in the second
approximated equation of motion gives us
\begin{align*}
    m(\alpha\ddot{\theta} + \ddot{\theta} l) + mg\theta &= 0\\
    \ddot{\theta}(l + \alpha) + g\theta &= 0\\
    \ddot{\theta} + \frac{g}{l'}\theta &= 0
\end{align*}
We see that the last equation is the differential equation which governs a
small oscillation pendulum of length $l'$ where $l'$ is
\begin{align*}
    l' = l + \frac{mg}{2k}
\end{align*}
\end{itemize}
\end{proof}

\cleardoublepage
\begin{proof}{\textbf{1.10}}
Let the force between two charged particles in motion be
\begin{align*}
    F = \frac{ee'}{r^2}\bigg[1 + \frac{r\ddot{r}}{c^2} - \frac{\dot{r}^2}{2c^2}\bigg]
    = ee'\bigg[\frac{1}{r^2} + \frac{\ddot{r}}{rc^2} - \frac{\dot{r}^2}{2r^2c^2}\bigg]
\end{align*}
Let us define the generalised potential as follows
\begin{align*}
    U = ee'\bigg[\frac{1}{r} + \frac{\dot{r}^2}{2rc^2}\bigg]
\end{align*}
Then we see that $F$ can be written as 
\begin{align*}
    F &= -\pdv{U}{r} + \dv{t}(\pdv{U}{\dot{r}})\\
    &= ee'\bigg[\bigg(\frac{1}{r^2} + \frac{\dot{r}^2}{2r^2c^2}\bigg)
    + \bigg(\frac{\ddot{r}}{rc^2} - \frac{\dot{r}^2}{r^2c^2}\bigg)\bigg]\\
    &= ee'\bigg[\frac{1}{r^2} + \frac{\ddot{r}}{rc^2} - \frac{\dot{r}^2}{2r^2c^2}\bigg]
\end{align*}
So $U$ is the generalised potential associated to $F$.
\\
Now, if a charge is in the presence of another charge held fixed at the origin,
then the force $F$ is the force between them, but $r$ is the polar radial
coordinate of the free charge. Then the kinetic energy of the system is
\begin{align*}
    T = \frac{m}{2}(\dot{r}^2 + r^2\dot{\theta}^2)
\end{align*}
Where $m$ is the mass of the charge. Then the Lagrangian of the system is
\begin{align*}
    L = T - U = \frac{m}{2}(\dot{r}^2 + r^2\dot{\theta}^2)
    - ee'\bigg[\frac{1}{r} + \frac{\dot{r}^2}{2rc^2}\bigg]
\end{align*}
Therefore the Lagrange's equations are
\begin{align*}
    \dv{t}(\pdv{L}{\dot{r}}) - \pdv{L}{r} &= 0\\
    \dv{t}(m\dot{r} - \frac{ee'\dot{r}}{rc^2})
    - \bigg[mr\dot{\theta}^2 + ee'\bigg[\frac{1}{r^2}
    + \frac{\dot{r}^2}{2r^2c^2}\bigg]\bigg] &= 0\\
    m\ddot{r} - ee'\bigg[\frac{\ddot{r}}{rc^2} - \frac{\dot{r}^2}{r^2c^2}\bigg]
    - \bigg[mr\dot{\theta}^2 + ee'\bigg[\frac{1}{r^2}
    + \frac{\dot{r}^2}{2r^2c^2}\bigg]\bigg] &= 0\\
    m(\ddot{r} - r\dot{\theta}^2) - ee'\bigg[
    \frac{1}{r^2} + \frac{\ddot{r}}{rc^2} - \frac{\dot{r}^2}{2r^2c^2}
    \bigg] &= 0
\end{align*}
And
\begin{align*}
    \dv{t}(\pdv{L}{\dot{\theta}}) - \pdv{L}{\theta} &= 0\\
    mr^2\dot{\theta} &= 0
\end{align*}
But, we see from the last equation that $\dot{\theta} = 0$ then the first
equation of motion becomes 
\begin{align*}
    m\ddot{r} &= ee'\bigg[
    \frac{1}{r^2} + \frac{\ddot{r}}{rc^2} - \frac{\dot{r}^2}{2r^2c^2}
    \bigg]
\end{align*}
\end{proof}

\cleardoublepage
\begin{proof}{\textbf{1.16}}
Let Bateman's Lagrangian
\begin{align*}
    L = e^{\lambda t}\bigg(\frac{m}{2}\dot{x}^2 - \frac{m\omega^2}{2}x^2\bigg)
\end{align*}
Then Lagrange's equation for this Lagrangian is
\begin{align*}
    \dv{t}(\pdv{L}{\dot{x}}) - \pdv{L}{x} &= 0\\
    \dv{t}(e^{\lambda t}m\dot{x}) - e^{\lambda t}m\omega^2x&= 0\\
    \lambda e^{\lambda t}m\dot{x} + e^{\lambda t}m\ddot{x}
    - e^{\lambda t}m\omega^2x&= 0\\
    \ddot{x} + \lambda \dot{x} - \omega^2x&= 0
\end{align*}
We see that this is the equation of motion of a damped harmonic oscillator.
\\
Let us define $q = e^{\lambda t/2}x$ then we see that $q^2 = e^{\lambda t}x^2$,
also, from the derivative of $q$ with respect to time we see that
\begin{align*}
    \dot{q} &= \frac{\lambda}{2}e^{\lambda t /2}x + e^{\lambda t /2}\dot{x}\\
    e^{\lambda t /2}\dot{x} &= \dot{q} - \frac{\lambda}{2}e^{\lambda t /2}x\\
    e^{\lambda t /2}\dot{x} &= \dot{q} - \frac{\lambda}{2}q\\
    e^{\lambda t}\dot{x}^2 &= \bigg(\dot{q} - \frac{\lambda}{2}q\bigg)^2
\end{align*}
Then the Lagrangian $L$ becomes
\begin{align*}
    L &= e^{\lambda t}\frac{m}{2}\dot{x}^2 - e^{\lambda t}\frac{m\omega^2}{2}x^2
    = \frac{m}{2}\bigg(\dot{q} - \frac{\lambda}{2}q\bigg)^2
    - \frac{m\omega^2}{2}q^2
\end{align*}
Hence the Lagrangian doesn't have an explicit time dependence.
\end{proof}

\cleardoublepage
\begin{proof}{\textbf{1.18}}
\begin{itemize}
\item [(a)] Let us consider Atwood's machine in terms of $x = x_1$ with a
massive pulley and a massive string with uniform linear mass density $\lambda$.
\\
The kinetic energy of the system is then given by
\begin{align*}
    T = \frac{1}{2}(m_1 + m_2 + m_s)\dot{x}^2 + \frac{1}{2} I \omega^2
\end{align*}
Where the first term takes into account the masses $m_1$, $m_2$ and the mass
of the string $m_s$. The second term is the rotational kinetic energy because
of the massive pulley, but we know that $\omega = \dot{x}/R$ where $R$ is the
radius of the pulley, then
\begin{align*}
    T = \frac{1}{2}(m_1 + m_2 + m_s + I/R^2)\dot{x}^2
\end{align*}
On the other hand, the potential energy because of the masses $m_1$ and $m_2$
is $-m_1gx + m_2gx$ since when one mass goes up the other goes down by the same
amount.
\\
Now, for the string, let $l$ be the length of the hanging part of the string,
then the mass of the string on the left side is
$\lambda x_1 = \lambda x$ and the mass on the right side is then
$\lambda (l - x)$. Taking into account the displacements of the centers
of mass of each side we get that the total potential energy is
\begin{align*}
    V &= -m_1gx + m_2gx - \lambda xg\frac{x}{2} - \lambda (l - x)g\frac{(l - x)}{2}\\
    &= (m_2 - m_1)gx - \frac{\lambda g}{2}x^2 - \frac{\lambda g}{2} (l - x)^2
\end{align*}
Therefore the Lagrangian of the system is
\begin{align*}
        L &= \frac{1}{2}(m_1 + m_2 + m_s + I/R^2)\dot{x}^2
        + (m_1 - m_2)gx + \frac{\lambda g}{2}x^2 + \frac{\lambda g}{2} (l - x)^2
\end{align*}

\item [(b)] Lagrange's equation for $x$ is then
\begin{align*}
    \dv{t}(\pdv{L}{\dot{x}}) - \pdv{L}{x} &= 0\\
    \dv{t}((m_1 + m_2 + m_s + I/R^2)\dot{x}) - 
    (m_1 - m_2)g - \lambda gx + \lambda g (l - x) &= 0\\
    (m_1 + m_2 + m_s + I/R^2)\ddot{x} - 
    (m_1 - m_2)g - \lambda gx + \lambda g (l - x) &= 0
\end{align*}
\item [(c)] Let $m_1 = m_2 = 0$ then the equation of motion becomes
\begin{align*}
    (m_s + I/R^2)\ddot{x} - \lambda gx + \lambda g (l - x) &= 0\\
    (m_s + I/R^2)\ddot{x} - 2\lambda gx + \lambda g l &= 0\\
    \ddot{x} - b^2x + \frac{b^2l}{2} &= 0
\end{align*}
Where $b = \sqrt{2\lambda g/(m_s + I/R^2)}$. We know that the solution to this
differential equation is
\begin{align*}
    x(t) = c_1 e^{bt} + c_2 e^{-bt} + \frac{l}{2}
\end{align*}
From the initial conditions $x(0)=x_0$ and $\dot{x}(0) = 0$ we can determine
$c_1$ and $c_2$, so we get that
\begin{align*}
    c_1 + c_2 = x_0 - \frac{l}{2}
\end{align*}
And from $\dot{x}(t) = c_1be^{bt} - c_2be^{-bt}$ we get that $0 = c_1 - c_2$ so
\begin{align*}
    c_1 = c_2 = \frac{1}{2}\bigg(x_0 - \frac{l}{2}\bigg)
\end{align*}
Therefore
\begin{align*}
    x(t) &= \frac{1}{2}\bigg(x_0 - \frac{l}{2}\bigg)(e^{bt} + e^{-bt}) + \frac{l}{2}\\
    &= \bigg(x_0 - \frac{l}{2}\bigg)\frac{e^{bt} + e^{-bt}}{2} + \frac{l}{2}\\
    &= \bigg(x_0 - \frac{l}{2}\bigg)\cosh bt + \frac{l}{2}
\end{align*}
If $x_0 = l/2$ we get that $x(t) = l/2$ which is the right physically expected
solution since each half of the string is hanging statically on each side of
the pulley.

\item [(d)] The exponential behaviour of the speed of the string for $t$ larger
than a few time constants is physically correct since we are considering the
case $m_1 = m_2 = 0$, so the string would slip and fall off the pulley increasing
its velocity infinitely.
\end{itemize}
\end{proof}

\cleardoublepage
\begin{proof}{\textbf{1.19}}
Let the following lagrangian
\begin{align*}
    L = \frac{1}{2}\frac{1}{1 + \lambda x^2}\bigg(\dot{x}^2 - \omega_0^2x^2\bigg)
    \quad\text{with}\quad \lambda > 0
\end{align*}
\begin{itemize}
\item [(a)] Then Lagrange's equation for $x$ is
\begin{align*}
    \dv{t}(\pdv{L}{\dot{x}}) - \pdv{L}{x} &= 0\\
    \dv{t}(\frac{\dot{x}}{1 + \lambda x^2})
    - \frac{1}{2}\bigg(
        -\frac{2\lambda x}{(1 + \lambda x^2)^2}(\dot{x}^2 - \omega_0^2x^2)
        -\frac{2\omega_0^2x}{1 + \lambda x^2}
    \bigg)
    &= 0\\
    \bigg(\frac{\ddot{x}}{1 + \lambda x^2}
    - \frac{2\lambda x \dot{x}^2}{(1 + \lambda x^2)^2}\bigg)
    + \bigg(
        \frac{\lambda x}{(1 + \lambda x^2)^2}(\dot{x}^2 - \omega_0^2x^2)
        + \frac{\omega_0^2x}{1 + \lambda x^2}
    \bigg)
    &= 0\\
    \bigg(\frac{\ddot{x}(1 + \lambda x^2)- 2\lambda x \dot{x}^2}{(1 + \lambda x^2)^2}\bigg)
    + \frac{\lambda x\dot{x}^2 - \omega_0^2\lambda xx^2 + \omega_0^2x(1 + \lambda x^2)}
    {(1 + \lambda x^2)^2}
    &= 0\\
    \ddot{x}(1 + \lambda x^2)- 2\lambda x \dot{x}^2
    + \lambda x\dot{x}^2 - \omega_0^2\lambda xx^2 + \omega_0^2x(1 + \lambda x^2)
    &= 0\\
    \ddot{x}(1 + \lambda x^2)- \lambda x \dot{x}^2 + \omega_0^2x &= 0
\end{align*}
\item [(b)] Let
\begin{align*}
    x = A\sin(\omega t + \delta)
\end{align*}
Then 
\begin{align*}
    \dot{x} = A\omega\cos(\omega t + \delta)
    \quad\text{and}\quad
    \ddot{x} = -A\omega^2\sin(\omega t + \delta)
\end{align*}
So replacing in the equation of motion we got from part (a) we see that
\begin{align*}
    &\ddot{x}(1 + \lambda x^2)- \lambda x \dot{x}^2 + \omega_0^2x = 0\\
    %
    &-A\omega^2\sin(\omega t + \delta)(1 + \lambda A^2\sin^2(\omega t + \delta))
    - \lambda A\sin(\omega t + \delta) A^2\omega^2\cos^2(\omega t + \delta)\\
    &\quad+ \omega_0^2A\sin(\omega t + \delta)
    = 0\\
    %
    &-A\omega^2\sin(\omega t + \delta)
    - \lambda A^3\omega^2\sin^3(\omega t + \delta)
    - \lambda A^3\omega^2\sin(\omega t + \delta)\cos^2(\omega t + \delta)\\
    &\quad+ \omega_0^2A\sin(\omega t + \delta)
    = 0
\end{align*}
We see that if $t = n\pi/\omega - \delta$ for $n = 0, 1, ...$ the equation is
satisfied, so let us assume $t \neq n\pi/\omega - \delta$ and hence
$\sin(\omega t + \delta) \neq 0$ then dividing by $\sin(\omega t + \delta)$
we get that
\begin{align*}
    -A\omega^2 - \lambda A^3\omega^2\sin^2(\omega t + \delta)
    - \lambda A^3\omega^2\cos^2(\omega t + \delta) + \omega_0^2A
    &= 0\\
    -A\omega^2 - \lambda A^3\omega^2(\sin^2(\omega t + \delta)
    + \cos^2(\omega t + \delta)) + \omega_0^2A
    &= 0\\
    \omega^2 + \lambda A^2\omega^2 - \omega_0^2
    &= 0
\end{align*}
So we see that this equation is satisfied if
$\omega^2 + \lambda A^2\omega^2 = \omega_0^2$. Therefore $\omega^2$ must have
the following value
\begin{align*}
    \omega^2 &= \frac{\omega_0^2}{1 + \lambda A^2}
\end{align*}
\end{itemize}
\end{proof}

\cleardoublepage
\begin{proof}{\textbf{1.20}}
From equation (1.162) and taking $V(r) = -\kappa/r$ we get that
\begin{align*}
    E &= \frac{m}{2}\bigg(\dv{r}{t}\bigg)^2 + \frac{l^2}{2mr^2} - \frac{\kappa}{r}
\end{align*}
So
\begin{align*}
    \frac{m}{2}\bigg(\dv{r}{t}\bigg)^2
    &= E + \frac{\kappa}{r} - \frac{l^2}{2mr^2}\\
    \dv{r}{t}
    &= \sqrt{\frac{2}{m}}\sqrt{ E + \frac{\kappa}{r} - \frac{l^2}{2mr^2}}
\end{align*}
Now, we integrate from $t = 0$ where $r = r_{min}$ (since the planet is at the
perihelion) to $t$ and $r$
\begin{align*}
    \int_0^t dt &= \sqrt{\frac{m}{2}}\int_{r_{min}}^r 
    \frac{dr}{\sqrt{ E + \frac{\kappa}{r} - \frac{l^2}{2mr^2}}}
\end{align*}
Also, we see that
\begin{align*}
    t &= \sqrt{\frac{m}{2}}\int_{r_{min}}^r 
    \frac{dr}{\sqrt{\frac{\kappa}{r^2}
    \big(\frac{Er^2}{\kappa} + r - \frac{l^2}{2m\kappa}\big)}}\\
    &= \sqrt{\frac{m}{2\kappa}}\int_{r_{min}}^r 
    \frac{rdr}{\sqrt{\frac{Er^2}{\kappa} + r - \frac{l^2}{2m\kappa}}}\\
    &= \sqrt{\frac{m}{2\kappa}}\int_{r_{min}}^r 
    \frac{rdr}{\sqrt{\frac{Er^2}{\kappa} + r - \frac{p}{2}}}
\end{align*}
Where $p = l^2/m\kappa$. We know as well that $e = \sqrt{1 + 2El^2/m\kappa^2}$
so
\begin{align*}
    -\frac{m}{2l^2}(1 - e^2) = \frac{E}{\kappa^2}
\end{align*}
But also since $a = p(1 - e^2)^{-1}$ we get that 
\begin{align*}
    \frac{E}{\kappa} &= -\frac{m\kappa}{2l^2}(1 - e^2)
    = -\frac{1}{2p}(1 - e^2)
    = -\frac{1}{2a}
\end{align*}
then
\begin{align*}
    t &= \sqrt{\frac{m}{2\kappa}}\int_{r_{min}}^r 
    \frac{rdr}{\sqrt{r -\frac{r^2}{2a} - \frac{a(1- e^2)}{2}}}
\end{align*}
Where we also used that $p = a(1 - e^2)$.
\\
Now, let us define $\psi$ as $r = a(1 - e\cos\psi)$ so $dr = ea\sin\psi d\psi$.
Then, replacing in the integral we get that
\begin{align*}
    t &= \sqrt{\frac{m}{2\kappa}}\int_{0}^\psi 
    \frac{ea^2(1 - e\cos\psi)\sin\psi d\psi}
    {\sqrt{a(1 - e\cos\psi) -\frac{a(1 - e\cos\psi)^2}{2} - \frac{a(1- e^2)}{2}}}
    \\
    &= \sqrt{\frac{m}{2\kappa}}\int_{0}^\psi 
    \frac{e\sqrt{a^3}(1 - e\cos\psi)\sin\psi d\psi}
    {\sqrt{1 - e\cos\psi - \frac{1 - 2e\cos\psi + e^2\cos^2\psi}{2}
    - \frac{1- e^2}{2}}}
    \\
    &= \sqrt{\frac{ma^3}{2\kappa}}\int_{0}^\psi 
    \frac{e(1 - e\cos\psi)\sin\psi d\psi}
    {\sqrt{1 - e\cos\psi - \frac{1}{2} + e\cos\psi - \frac{e^2\cos^2\psi}{2}
    - \frac{1}{2} + \frac{e^2}{2}}}
    \\
    &= \sqrt{\frac{ma^3}{2\kappa}}\int_{0}^\psi 
    \frac{e(1 - e\cos\psi)\sin\psi d\psi}
    {\sqrt{\frac{e^2}{2} - \frac{e^2\cos^2\psi}{2}}}
    \\
    &= \sqrt{\frac{ma^3}{2\kappa}}\int_{0}^\psi 
    \frac{\sqrt{2}(1 - e\cos\psi)\sin\psi d\psi}{\sqrt{(1 - \cos^2\psi)}}
    \\
    &= \sqrt{\frac{ma^3}{\kappa}}\int_{0}^\psi 
    \frac{(1 - e\cos\psi)\sin\psi d\psi}{\sin\psi}\\
    &= \sqrt{\frac{ma^3}{\kappa}}\int_{0}^\psi(1 - e\cos\psi) d\psi\\
    &= \sqrt{\frac{ma^3}{\kappa}}(\psi - e\sin\psi)
\end{align*}
Therefore naming $\omega = \sqrt{\kappa/ma^3}$ we get that
\begin{align*}
    \omega t = \psi - e\sin\psi
\end{align*}
\end{proof}

\cleardoublepage
\begin{proof}{\textbf{Exercise 1.7.1}}
Let a circular orbit such that $r = r_0$ then energy of the system is
\begin{align*}
    E_0 = \frac{1}{2}m\dot{r}^2 + V_{\text{eff}}(r)
\end{align*}
But since $r$ is constant then we have that $\dot{r} = 0$ and hence
\begin{align*}
    E_0 = V_{\text{eff}}(r_0)
\end{align*}
Then taking the time derivative we get that $V'_{\text{eff}}(r_0) = 0$
since $E_0$ is constant.
\\
But also we know that $V_{\text{eff}}(r) = l^2/2mr^2 + V(r)$ hence taking the
derivative and making it equal to 0 we get that
\begin{align*}
    -\frac{l^2}{mr^3} + V'(r) &= 0
\end{align*}
or replacing $r = r_0$ we get that $V'(r_0) = l^2/mr_0^3$.
\\
Finally, taking the second derivative of $V_{\text{eff}}(r)$ we get that
\begin{align*}
    V''_{\text{eff}}(r_0) &= \frac{3l^2}{mr_0^4} + V''(r_0)
    = \frac{3}{r_0}\frac{l^2}{mr_0^3} + V''(r_0)
    = \frac{3}{r_0}V'(r_0) + V''(r_0)
\end{align*}
\end{proof}

\cleardoublepage
\begin{proof}{\textbf{Exercise 1.7.2}}
\begin{itemize}
\item [(1)] Let $0 < e < 1$, the equation for $r$ is
\begin{align*}
    r = \frac{l^2}{m\kappa}\frac{1}{(1 + e\cos\phi)}
\end{align*}
Then the maximum and minimum of $r$ happen for $\phi = \pi$ and $\phi = 0$
respectively. For $\phi = \pi$ we have that $\cos\pi = -1$ and hence
\begin{align*}
    r_{\text{max}} = \frac{l^2}{m\kappa}\frac{1}{1 - e} = p(1 - e)^{-1}
\end{align*}
And for $\phi = 0$ we get that $\cos 0 = 1$ so
\begin{align*}
    r_{\text{min}} = \frac{l^2}{m\kappa}\frac{1}{1 + e} = p(1 + e)^{-1}
\end{align*}
Where $p = l^2/m\kappa$. Also, since the semi-major axis of an ellipse is
given by $a = (r_{\text{max}} + r_{\text{min}})/2$ we get that
\begin{align*}
    a = \frac{p(1 - e)^{-1} + p(1 + e)^{-1}}{2}
    = p\frac{1 + e + 1 - e}{2(1 - e)(1 + e)}
    = p(1 - e^2)^{-1}
\end{align*}

\item [(2)] We know that $dA = \frac{1}{2}r^2d\phi$ then dividing by $dt$
\begin{align*}
    \dv{A}{t} &= \frac{1}{2}r^2\dv{\phi}{t}
\end{align*}
But we know that $l = mr^2\dot\phi$ then
\begin{align*}
    \dv{A}{t} &= \frac{1}{2m}mr^2\dv{\phi}{t} = \frac{l}{2m}
\end{align*}
Which is Kepler's second law.

\item [(3)] From Kepler's second law we know that $dA = (l/2m)dt$ so by
integration we get that
\begin{align*}
    \frac{l}{2m}\int_0^\tau dt &= \int_0^A dA\\
    \tau &= \frac{2m}{l}A
\end{align*}
Replacing $A = \pi ab$ and $b = a\sqrt{1 - e^2}$ we get that
\begin{align*}
    \tau &= \frac{2m}{l}\pi ab = \frac{2\pi m a^2}{l}\sqrt{1 - e^2}
\end{align*}
Also from $a = p(1 -e^2)^{-1}$ we know that $1 - e^2 = p/a$ hence
\begin{align*}
    \tau &= \frac{2\pi m a^2}{l}\sqrt{\frac{p}{a}}
    = \frac{2\pi m a^2}{l}\sqrt{\frac{l^2}{m\kappa a}}
    = \frac{2\pi a^2}{\sqrt{a}}\sqrt{\frac{m}{\kappa}}
    = 2\pi a^{3/2}\sqrt{\frac{m}{\kappa}}
\end{align*}

\end{itemize}
\end{proof}
\end{document}